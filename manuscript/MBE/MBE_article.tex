\documentclass[nogrid]{MBE_article}%

\usepackage{url}

\jshort{mst}

\volname{}

\jvolume{0}

\jvol{}

\jissue{0}

\pubyear{2016}

\mstype{Letter}

\artid{012}

\access{Advance Access publication November 2, 2016}


\begin{document}

\title[BuddySuite]{BuddySuite: Command-line toolkits for manipulating sequences, alignments, and phylogenetic trees}


\author[Bond et al.]{Stephen R. \surname{Bond}, Karl E.
Keat, Sofia N. Barreira, and Andreas D. Baxevanis$^{\ast}$}

\address{Computational and Statistical Genomics Branch, Division of Intramural Research, National Human Genome Research Institute, National Institutes of Health, 50 South Drive, Bethesda, MD, USA, 20892}


\history{Received 12 November 2016}

\coresp{E-mail: andy@mail.nih.gov}


\editor{}

\abstract{The ability to manipulate sequence, alignment, and phylogenetic tree files has become an increasingly important skill in the life sciences, whether to generate summary information or to prepare data for further downstream analysis. The command line can be an extremely powerful environment for interacting with these resources, but only if the user has the appropriate general-purpose tools on hand. BuddySuite is a collection of four independent yet interrelated command-line toolkits that facilitate each step in the workflow of sequence discovery, curation, alignment, and phylogenetic reconstruction. Most common sequence, alignment, and tree file formats are automatically detected and parsed, and over 100 tools have been implemented for manipulating these data. The project has been engineered to easily accommodate the addition of new tools, it is written in the popular programming language Python, and is hosted on the Python Package Index and GitHub to maximize accessibility. Documentation for each BuddySuite tool, including usage examples, is available at http://tiny.cc/buddysuite\_wiki. All software is open source and freely available through http://research.nhgri.nih.gov/software/BuddySuite}

\keyword{software, command line, sequence, alignment, phylogenetic tree, Python}


\maketitle


\section{{Introduction}\label{sec:Intro}}
Manipulation of biological sequence data is now a routine task within the life sciences, performed not just by bioinformaticians but also by `bench biologists' who are becoming increasingly savvy in applying computational methods to their own work. While there are excellent graphical platforms for organizing, visualizing, and manipulating sequence data, it can be advantageous to interact with text files directly from the command line. Common tasks may include searching for specific records in a file, extracting subsequences, converting between formats, identifying motifs, or stripping poorly aligned regions from a multiple sequence alignment. While all of these can be accomplished with standard UNIX commands or existing open source software, combining tasks into a pipeline may require a series of intermediate files, complicated command-line operations, and moving data between standalone tools or online services. As an alternative we present BuddySuite, a comprehensive set of general-purpose command-line tools for manipulating sequence, alignment, and phylogenetic tree data that can be joined into reproducible workflows using a simple unified syntax.

The European Molecular Biology Open Software Suite (EMBOSS) \cite{Rice:2000wr} and Biopieces are the most comprehensive general-purpose open-source bioinformatics toolkits currently available for the command line. While both are excellent software packages, BuddySuite includes a number of new features we believe will benefit biologists. A notable change is our switch away from the `one program per function' paradigm that EMBOSS and Biopieces employ; both of these packages contain over 200 separate programs, thus occupying a considerable namespace on a user's system. For comparison, the 105 function implemented in BuddySuite version 1.2+ are all contained within just four modules, SeqBuddy, AlignBuddy, PhyloBuddy, and DatabaseBuddy, with each module being responsible for a specific data type (i.e., sequence, alignment, phylogenetic tree, or online sequence database queries, respectively). The first three modules rely on flags to differentiate among the many individual tools, which facilitates a unified and predictable interface. DatabaseBuddy, on the other hand, runs primarily as a `live shell' where users interactively search and download sequence data stored in the NCBI, UniProt, and Ensembl public databases. File format detection is fully automated; any number of sequence, alignment, or phylogenetic tree files can be passed into their respective BuddySuite program, in any combination of supported formats, and the records will be parsed seamlessly (see table \ref{table:formats} for a list of supported formats). This is particularly useful when using the BuddySuite modules to call third party alignment or phylogenetic inference programs, as any idiosyncratic format conversions are handled without the need of additional input from the user. All output is printed directly to the terminal window by default and each module adheres to the UNIX convention of accepting piped data, allowing individual tools to be `daisy-chained' into more complex workflows (illustrated further in the `Use-case examples' section below)

One of the greatest advantages BuddySuite has over other programs is its handling of sequence feature annotation. Rich flat file specifications like GenBank and EMBL support annotation, but this information is generally discarded by EMBOSS and Biopieces is unable to create new records in these formats. BuddySuite modules are aware of features in the sequence files they process and will update those annotations if a sequence is modified. For example, when a GenBank cDNA file is translated to protein, the relative positions of each feature is scaled by one third to account for the conversion of codons to amino acids. Similarly, if those protein sequences are passed into a supported multiple sequence alignment program, such as MAFFT, the feature positions will be adjusted to account for gaps.

\begin{table}[!t]
\tableparts{\caption{File format support for reading (R) and writing (W) provided by each BuddySuite module.\label{table:formats}}}
      {\tabcolsep=4pt\begin{tabular}{@{\extracolsep{\fill}}llll@{}}
        \toprule
        Format							& SeqBuddy  					& AlignBuddy   					& PhyloBuddy
        \\\colrule
        Clustal 						& R \& W\textsuperscript{\dag} 	& R \& W						& None \\ 
        EMBL\textsuperscript{\ddag} 	& R \& W						& R\textsuperscript{\dag}/ W	& None \\
        FASTA 							& R \& W						& R\textsuperscript{\dag}/ W	& None \\
        GenBank\textsuperscript{\ddag} 	& R \& W						& R\textsuperscript{\dag}/ W 	& None \\
        Nexus 							& R \& W\textsuperscript{\dag}	& R \& W						& R \& W \\ 
        Newick 							& None							& None							& R \& W \\ 
        NeXML							& None							& None							& R \& W \\
        PHYLIP (interleaved)			& R \& W\textsuperscript{\dag} 	& R \& W						& None \\
        PHYLIP (sequential)				& R \& W\textsuperscript{\dag} 	& R \& W						& None \\	
        SeqXML							& R \& W						& None							& None \\ 
        Stockholm						& R \& W\textsuperscript{\dag} 	& R \& W						& None \\ 
        SWISS-PROT\textsuperscript{\ddag} & R only						& None							& None
        \\\botrule
      \end{tabular}}
{\tablenote{\textsuperscript{\dag}All sequences must be the same length \\
      		  \textsuperscript{\ddag}Supports rich sequence annotation}}
\end{table}


\section{Use-case examples}
The examples highlighted in this section are intended to illustrate common features of the BuddySuite modules. For extended documentation please refer to the public wiki (http://tiny.cc/buddysuite\_wiki), where each tool is explained in detail.

BuddySuite modules are executed from the command line using the following generalized syntax:

\smallskip

{\small
\begin{verbatim}
 $: module file(s) <cmd> <args> <modifiers>
\end{verbatim}
}
\smallskip

Any number of files may be passed into the module, followed by the flag that denotes a desired command and any additional arguments that command requires. As a specific example, the following would accept two sequence files, one in FASTA format and the other in GenBank format, then delete sequences larger than 300 residues (note that module names have been shortened to sb, alb, and pb for SeqBuddy, AlignBuddy, and PhyloBuddy, respectively):

\smallskip

{\small
\begin{verbatim}
 $: sb seqs1.gb seqs2.fa --delete_large 300
\end{verbatim}
}
\smallskip

Whenever multiple file specifications are combined like this, the rightmost file will determine the final output format (in this case, FASTA), although this behaviour may be overridden with the `-{}-output' modifier. The command below would yield records in GenBank format:

\smallskip

{\small
\begin{verbatim}
 $: sb seqs1.gb seqs2.fa --delete_large 300
    --output genbank
\end{verbatim}
}
\smallskip

Modifiers like \mbox{`-{}-output'} are used sparingly in BuddySuite and only when their effects are intuitively applicable across most tools (e.g., \mbox{`-{}-quiet'} execution or to rewrite files \mbox{`-{}-in\_place'}).

Complex workflows can also be built from the BuddySuite modules using the pipe character:

\smallskip
{\small
\begin{verbatim}
 $: sb  cDNA_seqs.fa --transmembrane_domains |
    sb  --pull_records "TMD4" |
    sb  --translate |
    alb --generate_alignment mafft |
    alb --extract_feature_sequences "TMD2:TMD4" |
    pb  --generate_tree raxmlHPC-SSE3 |
    pb  --root
\end{verbatim}
}

\smallskip

In the example above, SeqBuddy processes a set of homologous cDNAs and transmits them to the TopCons server \cite{Tsirigos:2015eo} to predict transmembrane domains (TMD). Once the results have been automatically retrieved from the server, the sequences are annotated and those that include `TMD4' are retained for further analysis (note that SeqBuddy recasts the format to GenBank when applying new sequence features). The cDNAs are translated to protein, a multiple sequence alignment is generated, the section of the alignment spanning the second through forth TMD is extracted, and then phylogenetic inference software is called via PhyloBuddy to generate a tree. Finally, the tree is rooted at its midpoint.

To piece this workflow together without BuddySuite, a user may identify TMDs with a command-line version of software like TMHMM \cite{Krogh:2001bv} or access the TOPCONS server directly from a web browser, and then parse the output files with awk or excel to create a list of appropriate sequence IDs. After pulling those sequences from the original file and translating them with seqret and transeq (from EMBOSS), they would need to be saved as a FASTA file before calling MAFFT to generate a multiple sequence alignment. Extracting the alignment columns covering TMD2 through TMD4 would require manual inspection or a custom script to match the location of each TMD from the TMHMM or TOPCONS results to the alignment. Once identified, seqret could be called again to pull the correct regions from the alignment and create a new file, before finally running RAxML to infer a phylogenetic tree and root it. While this pipeline is valid, BuddySuite offers a solution with fewer steps, fewer intermediary files, less manual intervention, and a consistent syntax. 


\section{Installation}
The BuddySuite libraries have been written in Python 3 for use on all major operating systems (Windows 7+, Mac OSX, and Linux). Stable release versions of BuddySuite can be installed directly from the Python Package Index (PyPI) using the popular package manager `pip', and the most recent development version is continually available from GitHub. Dependencies have been limited to packages available through PyPI to simplify installation, although a number of optional third-party programs can also be accessed through BuddySuite; these include BLAST for comparing sequences, multiple sequence alignment packages like MAFFT, and phylogenetic inference packages like RAxML. These programs (itemized in table \ref{table:software}) are not necessary for the general operation of the BuddySuite modules, so installation is discretionary.

Users are also encouraged to run the optional BuddySuite configuration script after installation:

\smallskip
{\small
\begin{verbatim}
 $: pip install buddysuite
 $: buddysuite -setup
\end{verbatim}
}
\smallskip

Doing so will create directories for caching data on the user's system and, to prevent possible IP blocking, will also register an email address for the tools that interact with public databases over an internet connection.

\begin{table}[!t]
\tableparts{\caption{List of optional third party software that BuddySuite programs can interact with.\label{table:software}}}
      {\tabcolsep=5pt\begin{tabular}{@{\extracolsep{\fill}}lll@{}}
      \toprule
	   						& Program								& Reference
      \\\colrule
      SeqBuddy				& BLAST 								& \cite{Camacho2009}
      \\\colrule
      AlignBuddy			& Clustal$\Omega$						& \cite{Sievers:2011fn} \\
        					& ClustalW2 							& \cite{Larkin:2007hz} \\
							& MAFFT 								& \cite{Katoh:2013hm} \\
							& MUSCLE 								& \cite{Edgar:2004bo} \\
							& PAGAN 								& \cite{Loytynoja:2012fy} \\
        					& PRANK 								& \cite{Loytynoja:2005cb}		
      \\\colrule
      PhyloBuddy			& FastTree								& \cite{Price:2010eg} \\
        					& RAxML 								& \cite{Stamatakis:2006de} \\
        					& PhyML 								& \cite{Guindon:2010gf}
      \\\botrule
      \end{tabular}}
{\tablenote{BuddySuite performs all necessary format conversion to call any of these tools and, where appropriate, returns the result in the same format as the input. This is particularly useful when creating multiple sequence alignments from annotated sequences in GenBank or EMBL format.}}
\end{table}

\section{Developers}
Looking forward, the modular nature of BuddySuite makes it particularly well suited to open-ended development. New tools are easily added to each existing module and new modules may eventually extend the suite to new data types. While we will continue to support and expand BuddySuite ourselves, we also strive to attract contribution from the broader community. To minimize barriers against community-driven development, the project is maintained on GitHub, has comprehensive unit test coverage of over 95\%, includes extensive and accessible documentation, and makes every effort to conform with open-source best practices \cite{Leprevost:2014gx,Seemann:2013ci}.

Instead of relying exclusively on active input from users, we have implemented an optional passive data collection system to monitor usage and report crashes. If activated, this software improvement program will periodically transmit anonymized data to the core developers. Plus, in the event of a crash, it will analyze the traceback and compare it against an online database of resolved issues before immediately informing the user if the problem is patched by a newer release.

The internal functionality of BuddySuite is also easily accessible by developers wishing to write third-party Python programs. Each module has a core `Buddy' class that automatically processes a variety of input types (including plain text, file paths, file handles, and lists of record objects), performs all necessary file format processing, and exposes methods for managing and writing the sequence or tree records. The functions in each library accept these `Buddy' objects as input and generally return them as output, thus providing a standardized application programing interface that facilitates interoperability among functions. Once installed, the BuddySuite libraries can be imported using conventional Python syntax.

\section{Conclusions}
BuddySuite has been designed from the ground up as an intuitive, extensible, and unified platform for routine command-line tasks performed on sequence, alignment, and phylogenetic tree files. This is the first time such a large suite of general-purpose bioinformatics utilities have been implemented purely in Python and packaged together under a flag-driven paradigm. Well-designed and actively supported open-source tools will be invaluable over the coming years as an increasing number of biologists turn to the command line to analyze their data. We hope that BuddySuite will be widely adopted by the community and, thanks to the passive data-collection features built into this project, we look forward to tailoring future development to the needs of our users.


\section{Acknowledgments}
This research was supported by the Intramural Research Program of the National Human Genome Research Institute, National Institutes of Health. We would also like to thank Drs. Maxence LeVasseur and Tyra Wolfsberg for their thoughtful feedback on this manuscript and the community members who contributed code to the project, big or small. It takes a village.

%\section{Application programing interface (API)}
%Each module has a core `Buddy' class that automatically handles a variety of input types (e.g., plain text, file paths or handles, or a list of Biopython objects), performs all necessary file format processing, and exposes methods for managing and writing the sequence or tree records. All of the API functions in each library accept these `Buddy' objects as input and generally return them as output, thus providing a standardizing interface that facilitates interoperability among functions. Once installed, the BuddySuite libraries can be imported into third-party scripts using standard Python syntax.


%\begin{figure*}[t]
%\begin{center}
%\includegraphics[height=0.4\textheight]{figures/figure_fixed.eps}
%\end{center}
%\caption{RefSeq records were identified in GenBank using the following query: ``Nematoda''[Organism] AND biomol\_mrna[PROP] AND refseq[filter]. The results were downloaded in GenBank format and subsamples of 10, 100, 1000, and 10000 records were used to test the runtime performance of the BuddySuite tools (excluding tools that depend on third-party programs or services). Each dot denotes a single BuddySuite tool, runtimes are the average of 10 replicates expressed in seconds (the y-axis is log-scale), and the subscript numbers below each jitter plot represents the size of the sequence, alignment, or tree file in bytes.}
%\label{fig:timeit}
%\end{figure*}


\bibliographystyle{natbib}%%%%natbib.sty
\bibliography{../references/refs}%%%refs.bib

\end{document}
